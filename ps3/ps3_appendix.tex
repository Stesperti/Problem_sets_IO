% !TeX root = main.tex

% class
	\documentclass[a4paper, 11pt]{article}

% packages
	% fonts
		\usepackage[T1]{fontenc}
		\usepackage[utf8]{inputenc}
		\usepackage{titlesec}
	% margins
		\usepackage[bottom = 2cm, top = 2cm, left = 2cm, right = 2cm]{geometry}
	% math
		\usepackage{amsmath}
		\usepackage{amsfonts}
		\usepackage{amsthm}
		\usepackage{bm}
		\usepackage{mathtools}
		\usepackage{dsfont}
	% references
		\usepackage[colorlinks = true, linkcolor = black, citecolor = black, urlcolor = gray]{hyperref}
		\usepackage{cleveref}
	% pictures
		\usepackage{graphicx}
		\usepackage{subcaption}
		\usepackage{float}
	% colors
		\usepackage{xcolor}
	% code
		\usepackage{listings}
	% other
		% boxes
			\usepackage{scalerel}
			\usepackage{adjustbox}
		% tables
			\usepackage{multirow}
		% itemize
			\usepackage{enumerate}
		% misc
			\usepackage{comment}	
			\usepackage{lipsum}
   \usepackage{natbib}
\bibliographystyle{plainnat}


% commands
	% math
		\newcommand{\prob}[1]{\mathbb{P}\left(#1\right)}
		\newcommand{\mean}[1]{\mathbb{E}\left[#1\right]}
		\newcommand{\var}[1]{\operatorname{Var}\left(#1\right)}
		\newcommand{\cov}[2]{\operatorname{Cov}\left(#1, #2\right)}
		\newcommand{\corr}[2]{\operatorname{Corr}\left(#1, #2\right)}
		\newcommand{\normal}[2]{\mathcal{N}\left(#1, #2\right)}
		\newcommand{\uniform}[1]{\operatorname{U}\left(#1\right)}
		\newcommand{\indic}{\mathds{1}}
		\newcommand{\real}{\mathbb{R}}
		\newcommand{\integer}{\mathbb{Z}}
		\renewcommand{\natural}{\mathbb{N}}
  \newcommand{\ra}{\rightarrow}
\newcommand{\iy}{\infty}
\newcommand{\mE}{\mathbb{E}}
\newcommand{\mP}{\mathbb{P}}
\newcommand{\mB}{\mathcal{B}}
\newcommand{\mL}{\mathbb{L}}
\newcommand{\mLL}{\mathbb{L}^2\left\left(0,T\right\right)}
\newcommand{\mR}{\mathbb{R}}
\newcommand{\mN}{\mathbb{N}}
\newcommand{\mF}{\mathcal{F}}
\newcommand{\mM}{\mathcal{M}}
\newcommand{\mW}{\mathcal{W}}
\newcommand{\mA}{\mathcal{A}}
\newcommand{\mez}{\frac{1}{2}}
\newcommand{\intT}{\int_{0}^{T}}
\newcommand{\dt}{\frac{\partial f}{\partial t}}
\newcommand{\dl}{\frac{\partial f}{\partial \lambda}}
\newcommand{\inti}{\int_{\left\left(0,\infty\right\right)}}
\newcommand{\intt}{\int_0^t}
\newcommand{\W}[1]{W\left\left({#1}_{k+1}\right\right)-W\left\left({#1}_k\right\right)}
\newcommand{\ninf}{n \ra \iy}
\newcommand{\nN}{n \in \mathbb{N}}
\newcommand{\Q}{Q_t}
\newcommand{\ig}{{\lfloor \gamma_k t \rfloor}}
\newcommand{\X}{\mathcal{X}}
\newcommand{\T}{\Theta}
\newcommand{\E}{\mathbb{E}}
\newcommand{\R}{\mathbb{R}}

% preamble
	% path for pictures
		\graphicspath{{./images/}}
	% adjust indentation
		\newlength\tindent
		\setlength{\tindent}{\parindent}
		\setlength{\parindent}{0pt}
		\renewcommand{\indent}{\hspace*{\tindent}}
	% set fbox dimension
		\setlength{\fboxsep}{6pt}
	% layout
		% math
			\newtheorem{thm}{Theorem}
			\newtheorem*{thm*}{Theorem}
			\crefname{thm}{Theorem}{Theorems}
			\theoremstyle{plain}
			\newtheorem{rem}{Remark}
			\newtheorem*{rem*}{Remark}
			\crefname{rem}{Remark}{Remarks}
			\newenvironment{solution}{\vspace{-0.3cm}\paragraph{\normalfont{\bfseries{Solution.}}}}{\hfill$\square$}
		% fonts
			\titleformat*{\section}{\Large\bfseries}
			\titleformat*{\subsection}{\large\bfseries}
		% bib
			\newcommand{\doi}[1]{\textsc{doi}: \href{https://doi.org/#1}{#1}}
		% title
			\makeatletter
				\def\maketitle{%
					\pagestyle{plain}
						\begin{flushleft}
							\normalfont{\small{%
								\@author \\
								\@date
							}}
						\end{flushleft}
						\begin{flushright}\vspace{-15mm}
						\includegraphics[height = 1.5cm]{\mainlogo}
						\end{flushright}
						\vspace{-0.1cm}
						\begin{center}\vspace{-5mm}
							\scshape{\Large{\bfseries{\maintitle}} \\
							\large \@title} \\
							\vspace{0.25cm}
							\rule{0.75\linewidth}{0.1mm}	
						\end{center}
					}
			\makeatother

\begin{document}

	  \author{Stefano Sperti}
	\date{\today}
	\title{Code Appendix}
        \def\maintitle{Problem set 3}
	\def\mainlogo{Latex/chicago_booth_logo.jpg}

	\maketitle
    \subsection{2}
Multinomial Logit Market Shares
$$
s_{ij}(\theta) 
= \Pr(d_{ij} = 1) 
= \int 
\frac{\exp\!\left(\beta_i x_j\right)}
{1 + \sum_{k \in \mathcal{J}} \exp\!\left(\beta_i x_k\right)}
\, f(\beta_i \mid \theta) \, \mathrm{d}\beta_i.
$$

Then, Let's differentiate with respect to the price variable. Then, for any product $m\in\mathcal J$,
$$
\frac{\partial s_{ij}(\theta)}{\partial p_m}
= \int \frac{\partial s_{ij}(\beta_i)}{\partial p_m}\, f(\beta_i\mid\theta)\,
\mathrm d\beta_i,
\qquad\text{with}\qquad
\frac{\partial s_{ij}(\beta_i)}{\partial p_m}
= s_{ij}(\beta_i)\,\big(\mathbf 1\{j=m\}-s_{im}(\beta_i)\big)\,\beta_i^{(p)}.
$$
Hence,
$$
\boxed{\;
\frac{\partial s_{ij}(\theta)}{\partial p_m}
= \int s_{ij}(\beta_i)\,\big(\mathbf 1\{j=m\}-s_{im}(\beta_i)\big)\,
\beta_i^{(p)}\, f(\beta_i\mid\theta)\,\mathrm d\beta_i.
\;}
$$

In particular,
$$
\frac{\partial s_{ij}(\theta)}{\partial p_j}
= \int s_{ij}(\beta_i)\big(1-s_{ij}(\beta_i)\big)\,\beta_i^{(p)}
\, f(\beta_i\mid\theta)\,\mathrm d\beta_i,
\quad
\frac{\partial s_{ij}(\theta)}{\partial p_m}
= -\int s_{ij}(\beta_i)s_{im}(\beta_i)\,\beta_i^{(p)}
\, f(\beta_i\mid\theta)\,\mathrm d\beta_i \;\;(m\neq j).
$$
\subsection*{4--5. Multinomial Logit Estimation (OLS and 2SLS)}

We estimate the plain multinomial logit model of demand by OLS, and then the same model by
two-stage least squares (2SLS), instrumenting for prices with the exogenous demand shifters $x$
and the excluded cost shifters $w$. The results are reported in Table~\ref{tab:exercise_4}.

\begin{table}[H]
    \centering
    \begin{table}
\caption{True parameters vs. OLS and IV-2SLS estimates}
\label{tab:exercise_2}
\begin{tabular}{lccc}
\toprule
 & True value & OLS & IV-2SLS (robust) \\
\midrule
lpha (price) & -2.000 & -1.353 & -1.405 \\
eta_1 (x) & 1.000 & 0.556 & 0.637 \\
eta_2 (is_satellite) & 4.000 & 0.042 & 0.120 \\
lpha (price) (s.e.) & NaN & 0.012 & 0.020 \\
eta_1 (x) (s.e.) & NaN & 0.028 & 0.034 \\
\\eta_2 (is_satellite) (s.e.) & NaN & 0.035 & 0.041 \\
\bottomrule
\end{tabular}
\end{table}

\end{table}

---

\subsection*{6. Nested Logit Model}

We next estimate a nested logit model by two-stage least squares, treating \textit{satellite} and
\textit{wired} as the two nests for the inside goods. The results are presented in
Table~\ref{tab:exercise_6}.

\begin{table}[H]
    \centering
    \begin{table}
\caption{Nested logit IV-2SLS estimates}
\label{tab:exercise_6}
\begin{tabular}{lccc}
\toprule
 & IV-2SLS \\
\midrule
$\alpha$ (price) & -0.258 \\
$\beta_1$ (x) & 0.348 \\
$\sigma_{satellite}$ & 0.814 \\
$\sigma_{wired} $& 0.860 \\
\bottomrule
\end{tabular}
\end{table}

\end{table}

\textbf{Why this model is misspecified:}  
[Insert explanation here—e.g., the nesting structure imposes substitution patterns that differ
from the true data-generating process.]

---

\subsection*{7. Elasticities and Diversion Ratios (Baseline Estimates)}

Table~\ref{tab:exercise_7} compares the estimated own-price elasticities to the true
own-price elasticities.  
Tables~\ref{tab:diversion_true} and~\ref{tab:diversion_est} report the true and estimated
diversion ratio matrices.

\begin{table}[H]
    \centering
    \begin{table}
\caption{Summary: True vs. Estimated Parameters and Elasticities}
\label{tab:exercise_7}
\begin{tabular}{rrr}
\toprule
Product & Mean True Elasticity & Mean Estimated Elasticity \\
\midrule
1 & -5.0870 & -1.2662 \\
2 & -5.1383 & -1.2851 \\
3 & -5.1099 & -1.2750 \\
4 & -5.0621 & -1.2655 \\
\bottomrule
\end{tabular}
\end{table}

\end{table}

\begin{table}[H]
    \centering
    \begin{table}
\caption{True Diversion Ratios}
\label{tab:diversion_true}
\begin{tabular}{rrrrr}
\toprule
Product 1 & Product 2 & Product 3 & Product 4 & Outside \\
\midrule
0.0000 & 0.2439 & 0.1408 & 0.1478 & 0.4675 \\
0.2412 & 0.0000 & 0.1383 & 0.1492 & 0.4714 \\
0.1465 & 0.1464 & 0.0000 & 0.2439 & 0.4632 \\
0.1446 & 0.1496 & 0.2345 & 0.0000 & 0.4714 \\
\bottomrule
\end{tabular}
\end{table}

\end{table}

\begin{table}[H]
    \centering
    \begin{table}
\caption{Estimated Diversion Ratios (Nested Logit, Market 1)}
\label{tab:diversion_est}
\begin{tabular}{lrrrr}
\toprule
 & 1 & 2 & 3 & 4 \\
\midrule
1 & 0.000 & 1.000 & 0.120 & 0.210 \\
2 & 1.000 & 0.000 & 0.120 & 0.210 \\
3 & 0.053 & 0.462 & 0.000 & 1.000 \\
4 & 0.053 & 0.462 & 1.000 & 0.000 \\
\bottomrule
\end{tabular}
\end{table}

\end{table}

---

\subsection*{8. Joint Demand and Supply Estimation}

We report the estimated demand parameters and standard errors under three specifications:
\begin{enumerate}
    \item Demand estimated alone;
    \item Demand and supply estimated jointly; and
    \item Demand estimated using the ``optimal IV''.
\end{enumerate}
Results appear in Table~\ref{tab:exercise_8}.

\begin{table}[H]
    \centering
    \begin{table}
\caption{Estimates of Demand Parameters and Standard Errors Across Specifications}
\label{tab:exercise_8}
\begin{tabular}{llll}
\toprule
Model & Demand Only & Joint Demand + Supply & Optimal IV \\
Parameter &  &  &  \\
\midrule
1 & 3.6764
(0.2832) & 3.6696
(0.2825) & 3.2241
(0.2064) \\
prices & -1.9705
(0.0935) & -1.9682
(0.0932) & -1.7969
(0.0643) \\
satellite & -0.9910
(0.2194) & -0.9859
(0.2188) & 0.0136
(0.0564) \\
x & 0.9617
(0.0471) & 0.9610
(0.0470) & 0.8289
(0.0319) \\
\bottomrule
\end{tabular}
\end{table}

\end{table}

---

\subsection*{9. Preferred Model: Elasticities and Diversion Ratios}

Using the preferred estimates from Section~8, we compare estimated and true own-price
elasticities (Table~\ref{tab:exercise_9}) and show the corresponding true and estimated
diversion ratios (Tables~\ref{tab:diversion_9_true}--\ref{tab:diversion_9_est}).

\begin{table}[H]
    \centering
    \begin{table}
\caption{True vs. Estimated Own-Price Elasticities at Market 1}
\label{tab:exercise_9}
\begin{tabular}{rrr}
\toprule
Product & True own-price elasticity & Estimated own-price elasticity \\
\midrule
1 & 0.0000 & -3.6855 \\
2 & 0.0000 & -2.2284 \\
3 & 0.0000 & -2.5992 \\
4 & 0.0000 & -3.0789 \\
\bottomrule
\end{tabular}
\end{table}

\end{table}

\begin{table}[H]
    \centering
    \input{ps3/latex/table_diversion_9_true}
\end{table}

\begin{table}[H]
    \centering
    \begin{table}
\caption{Estimated Diversion Ratios}
\label{tab:diversion_9_est}
\begin{tabular}{rrrr}
\toprule
0 & 1 & 2 & 3 \\
\midrule
0.4352 & 0.1956 & 0.1849 & 0.1844 \\
0.1859 & 0.4419 & 0.1861 & 0.1862 \\
0.1867 & 0.1979 & 0.4350 & 0.1804 \\
0.1861 & 0.1977 & 0.1806 & 0.4356 \\
\bottomrule
\end{tabular}
\end{table}

\end{table}
\subsection{10}
Suppose two of the four firms were to merge. Give a brief intuition for what theory tells us is
likely to happen to the equilibrium prices of each good j.

\subsection*{12}
comparing the (average across markets) predicted merger-induced price
changes for this merger 
\subsection*{13}
13. Thus far you have assumed that there are no “efficiencies” (reduction in costs) resulting from
the merger. Explain briefly why a merger-specific reduction in marginal cost could mean that
a merger is welfare-enhancing
















\end{document}



