% !TeX root = main.tex

% class
	\documentclass[a4paper, 11pt]{article}

% packages
	% fonts
		\usepackage[T1]{fontenc}
		\usepackage[utf8]{inputenc}
		\usepackage{titlesec}
	% margins
		\usepackage[bottom = 2cm, top = 2cm, left = 2cm, right = 2cm]{geometry}
	% math
		\usepackage{amsmath}
		\usepackage{amsfonts}
		\usepackage{amsthm}
		\usepackage{bm}
		\usepackage{mathtools}
		\usepackage{dsfont}
	% references
		\usepackage[colorlinks = true, linkcolor = black, citecolor = black, urlcolor = gray]{hyperref}
		\usepackage{cleveref}
	% pictures
		\usepackage{graphicx}
		\usepackage{subcaption}
		\usepackage{float}
	% colors
		\usepackage{xcolor}
	% code
		\usepackage{listings}
	% other
		% boxes
			\usepackage{scalerel}
			\usepackage{adjustbox}
		% tables
			\usepackage{multirow}
		% itemize
			\usepackage{enumerate}
		% misc
			\usepackage{comment}	
			\usepackage{lipsum}
   \usepackage{natbib}
\bibliographystyle{plainnat}


% commands
	% math
		\newcommand{\prob}[1]{\mathbb{P}\left(#1\right)}
		\newcommand{\mean}[1]{\mathbb{E}\left[#1\right]}
		\newcommand{\var}[1]{\operatorname{Var}\left(#1\right)}
		\newcommand{\cov}[2]{\operatorname{Cov}\left(#1, #2\right)}
		\newcommand{\corr}[2]{\operatorname{Corr}\left(#1, #2\right)}
		\newcommand{\normal}[2]{\mathcal{N}\left(#1, #2\right)}
		\newcommand{\uniform}[1]{\operatorname{U}\left(#1\right)}
		\newcommand{\indic}{\mathds{1}}
		\newcommand{\real}{\mathbb{R}}
		\newcommand{\integer}{\mathbb{Z}}
		\renewcommand{\natural}{\mathbb{N}}
  \newcommand{\ra}{\rightarrow}
\newcommand{\iy}{\infty}
\newcommand{\mE}{\mathbb{E}}
\newcommand{\mP}{\mathbb{P}}
\newcommand{\mB}{\mathcal{B}}
\newcommand{\mL}{\mathbb{L}}
\newcommand{\mLL}{\mathbb{L}^2\left\left(0,T\right\right)}
\newcommand{\mR}{\mathbb{R}}
\newcommand{\mN}{\mathbb{N}}
\newcommand{\mF}{\mathcal{F}}
\newcommand{\mM}{\mathcal{M}}
\newcommand{\mW}{\mathcal{W}}
\newcommand{\mA}{\mathcal{A}}
\newcommand{\mez}{\frac{1}{2}}
\newcommand{\intT}{\int_{0}^{T}}
\newcommand{\dt}{\frac{\partial f}{\partial t}}
\newcommand{\dl}{\frac{\partial f}{\partial \lambda}}
\newcommand{\inti}{\int_{\left\left(0,\infty\right\right)}}
\newcommand{\intt}{\int_0^t}
\newcommand{\W}[1]{W\left\left({#1}_{k+1}\right\right)-W\left\left({#1}_k\right\right)}
\newcommand{\ninf}{n \ra \iy}
\newcommand{\nN}{n \in \mathbb{N}}
\newcommand{\Q}{Q_t}
\newcommand{\ig}{{\lfloor \gamma_k t \rfloor}}
\newcommand{\X}{\mathcal{X}}
\newcommand{\T}{\Theta}
\newcommand{\E}{\mathbb{E}}
\newcommand{\R}{\mathbb{R}}

% preamble
	% path for pictures
		\graphicspath{{./images/}}
	% adjust indentation
		\newlength\tindent
		\setlength{\tindent}{\parindent}
		\setlength{\parindent}{0pt}
		\renewcommand{\indent}{\hspace*{\tindent}}
	% set fbox dimension
		\setlength{\fboxsep}{6pt}
	% layout
		% math
			\newtheorem{thm}{Theorem}
			\newtheorem*{thm*}{Theorem}
			\crefname{thm}{Theorem}{Theorems}
			\theoremstyle{plain}
			\newtheorem{rem}{Remark}
			\newtheorem*{rem*}{Remark}
			\crefname{rem}{Remark}{Remarks}
			\newenvironment{solution}{\vspace{-0.3cm}\paragraph{\normalfont{\bfseries{Solution.}}}}{\hfill$\square$}
		% fonts
			\titleformat*{\section}{\Large\bfseries}
			\titleformat*{\subsection}{\large\bfseries}
		% bib
			\newcommand{\doi}[1]{\textsc{doi}: \href{https://doi.org/#1}{#1}}
		% title
			\makeatletter
				\def\maketitle{%
					\pagestyle{plain}
						\begin{flushleft}
							\normalfont{\small{%
								\@author \\
								\@date
							}}
						\end{flushleft}
						\begin{flushright}\vspace{-15mm}
						\includegraphics[height = 1.5cm]{\mainlogo}
						\end{flushright}
						\vspace{-0.1cm}
						\begin{center}\vspace{-5mm}
							\scshape{\Large{\bfseries{\maintitle}} \\
							\large \@title} \\
							\vspace{0.25cm}
							\rule{0.75\linewidth}{0.1mm}	
						\end{center}
					}
			\makeatother

\begin{document}

	  \author{Stefano Sperti}
	\date{\today}
	\title{Code Appendix}
        \def\maintitle{Problem set 0}
	\def\mainlogo{chicago_booth_logo.jpg}

	\maketitle
\section{Part 0: Logit Inclusive Value}

Define the (log-sum-exp) inclusive value
\[
\mathrm{IV}(x) \;=\; \log\!\Bigg(\sum_{i=1}^N e^{x_i}\Bigg), \qquad x\in\mathbb{R}^N.
\]
Let \(S=\sum_{j=1}^N e^{x_j}\) and \(p_i = e^{x_i}/S\). Then
\[
\nabla \mathrm{IV}(x) = p \quad\text{and}\quad 
\nabla^2 \mathrm{IV}(x) = \operatorname{diag}(p) - pp^\top .
\]
For any \(v\in\mathbb{R}^N\),
\[
v^\top \nabla^2 \mathrm{IV}(x)\, v
= \sum_{i=1}^N p_i v_i^2 - \Big(\sum_{i=1}^N p_i v_i\Big)^2
= \operatorname{Var}_p(V) \;\ge\; 0,
\]
where \(V\) takes value \(v_i\) with probability \(p_i\). Hence \(\nabla^2 \mathrm{IV}(x)\) is positive semidefinite and \(\mathrm{IV}\) is convex.

\section{Part 2: Numerical integration}
\subsection*{2.7} The following tables present the results obtained from the code.
\begin{table}[htbp]
\centering
\begin{tabular}{ccccc}
Method & Value & Abs. error & Rel. error & N points\\
QuadGK, True Value & 0.555939 & - & - & —\\
MC 200 & 0.541772 & 0.014167 & 2.548265 & 200\\
MC 400 & 0.543203 & 0.012736 & 2.290979 & 400\\
GH 4 & 0.555916 & 2.3e-5 & 0.004225 & 4\\
GH 8 & 0.555939 & 0.0 & 1.1e-5 & 8\\
GH 12 & 0.555939 & 0.0 & 0.0 & 12\\
\end{tabular}

\caption{1D integration results: value, error, and number of points.}
\label{tab:results-1d}
\end{table}

\begin{table}[htbp]
\centering
\begin{tabular}{ccccc}
Method & Value & Abs. error & Rel. error & N points\\
QuadGK, True Value & 0.725886 & - & - &-\\
MC 200 & 0.703327 & 0.022559 & 3.107809 & 200\\
MC 400 & 0.711449 & 0.014437 & 1.988858 & 400\\
\end{tabular}

\caption{2D integration results: value, error, and number of points.}
\label{tab:results-2d}
\end{table}

\end{document}

